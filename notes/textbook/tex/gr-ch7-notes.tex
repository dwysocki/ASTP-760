\documentclass[gr-notes.tex]{subfiles}

\begin{document}

\setcounter{chapter}{6}

\chapter{Physics in a curved spacetime}

\setcounter{section}{5}

\section{Exercises}


\textbf{1}
If Equation 7.3 were the correct generalization of 7.1 in a curved spacetime, what are the implications? What would happen to the number of particles in a comoving volume of the fluid over time? May we experimentally distinguish between Equations 7.2 and 7.3?

The number of particles would change proportionally to the square of the Ricci scalar, which corresponds to the curvature of the manifold. Whether particles are created ($+$) or destroyed ($-$) would depend on the sign of $q$ in the equation.

We could set up some experiment which tests for a change in the number of particles in a moving fluid, in various gravitational fields, to verify whether the RHS of the equation is non-zero.




\textbf{2}
Compute $g^{\alpha\beta}$ for the line element given by Equation 7.8, to first order in $\phi$.

Based on the line element, we can infer that the metric is
%
\begin{align*}
  (g_{\alpha\beta}) &\underset{(t,x,y,z)}{\to}
  \mqty(\dmat[0]{-(1+2\phi),(1-2\phi),(1-2\phi),(1-2\phi)})
  \\
  (g^{\alpha\beta}) &\underset{(t,x,y,z)}{\to}
  \mqty(\dmat[0]{-(1+2\phi)^{-1},(1-2\phi)^{-1},(1-2\phi)^{-1},(1-2\phi)^{-1}})
  \approx
  \mqty(\dmat[0]{-(1-2\phi),(1+2\phi),(1+2\phi),(1+2\phi)})
\end{align*}



\textbf{3}
Calculate the Christoffel sybmols for the metric given by Equation 7.8, to first order in $\phi$, assuming $\phi = \phi(t,x,y,z)$.

I do the following with the assistance of the free computer algebra system \texttt{Maxima}. I used the exact form of the metric tensor, and then approximated the resulting Christoffel symbols to first order in $\phi$.

\begin{align*}
  \tensor{\Gamma}{^t_{t\alpha}} &=
  \pdv{\phi}{x^\alpha} \frac{1}{1 + 2\phi} \approx
  \pdv{\phi}{x^\alpha} (1 - 2\phi)
  &
  \tensor{\Gamma}{^\alpha_{t\alpha}} &=
 -\pdv{\phi}{t} \frac{1}{1 - 2\phi} \approx
 -\pdv{\phi}{t} (1 + 2\phi)
  \\
  \tensor{\Gamma}{^i_{tt}} &=
  \pdv{\phi}{x^i} \frac{1}{1 - 2\phi} \approx
  \pdv{\phi}{x^i} (1 + 2\phi)
  &
  \tensor{\Gamma}{^t_{ii}} &=
 -\pdv{\phi}{t} \frac{1}{1 + 2\phi} \approx
  \pdv{\phi}{t} (1 - 2\phi)
  \\
  \tensor{\Gamma}{^i_{jj}} =
 -\tensor{\Gamma}{^j_{ij}} =
 -\tensor{\Gamma}{^i_{ii}} &=
  \pdv{\phi}{x^i} \frac{1}{1 - 2\phi} \approx
  \pdv{\phi}{x^i} (1 + 2\phi)
\end{align*}


\textbf{5}

(a) In the case of a perfect fluid, verify that the spatial components of Equation 7.6 reduce to
%
\begin{displaymath}
  \dot{\boldsymbol{v}} +
  (\boldsymbol{v} \cdot \grad) \boldsymbol{v} +
  \grad p / \rho +
  \grad \phi =
  0
\end{displaymath}
%
in the Newtonian limit and in the weak-field regime (the metric given by Equation 7.8).

\begin{align*}
  T^{\mu\nu} &=
  (\rho + p) U^\mu U^\nu + p g^{\mu\nu}
  \\ &\approx
  \rho U^\mu U^\nu + p g^{\mu\nu}
  \\ &\approx
  m U^\mu (n U^\nu) + p g^{\mu\nu}
  \\
  T^{i\nu} &\approx
  m U^i (n U^\nu) + p g^{i\nu}
  \\
  \tensor{T}{^{i\nu}_{;\nu}} &\approx
  m [ U^i (n U^\nu) ]_{;\nu} + [ p g^{i\nu} ]_{;\nu} =
  m n U^\nu \tensor{U}{^i_{;\nu}} + g^{i\nu} p_{;\nu} = 0
  \\ \implies
  0 &=
  U^\nu \tensor{U}{^i_{;\nu}} + g^{i\nu} p_{;\nu} / \rho
  \\ &=
  U^\nu (\tensor{U}{^i_{,\nu}} + U^\lambda \tensor{\Gamma}{^i_{\lambda\nu}}) +
  g^{i\nu} p_{,\nu} / \rho
  \\ &=
  U^0 \tensor{U}{^i_{,0}} +
  U^j \tensor{U}{^i_{,j}} +
  U^\nu U^\lambda \tensor{\Gamma}{^i_{\lambda\nu}} +
  g^{i\nu} p_{,\nu} / \rho
  \\ &=
  \gamma \dv{\tau} (\gamma v^i) +
  \gamma v^j (\gamma v^i)_{,j} +
  U^\nu U^\lambda \tensor{\Gamma}{^i_{\lambda\nu}} +
  g^{ii} p_{,i} / \rho
  \\ &\approx
  \dv{v^i}{\tau} +
  v^j v^i_{,j} +
  (U^0)^2 \tensor{\Gamma}{^i_{00}} +
  (1 - 2\phi) p_{,i} / \rho
  \\ &\approx
  \dv{v^i}{\tau} +
  v^j v^i_{,j} +
  \phi_{,i} +
  p_{,i} / \rho
\end{align*}
%
rewriting this in vector form, we get the original equation.


(b) Now look at the time-component instead of the spatial component.

\begin{align*}
  T^{0\nu} &=
  (\rho + p) U^0 U^\nu + p g^{0\nu} \approx
  m U^0 (n U^\nu) + p g^{0\nu}
  \\
  \tensor{T}{^{0\nu}_{;\nu}} &=
  m [U^0 (n U^\nu)]_{;\nu} + [p g^{0\nu}]_{;\nu} =
  m n U^\nu \tensor{U}{^0_{;\nu}} + g^{00} p_{,\nu} =
  0
  \\ \implies
  0 &=
  U^\nu \tensor{U}{^0_{;\nu}} + g^{00} \dot{p} / \rho
  \\ &=
  U^\nu (\tensor{U}{^0_{,\nu}} + U^\lambda \tensor{\Gamma}{^0_{\lambda\nu}}) +
  g^{00} \dot{p} / \rho
  \\ &=
  U^0 \tensor{U}{^0_{,0}} +
  U^i \tensor{U}{^0_{,i}} +
  U^\nu U^\lambda \tensor{\Gamma}{^0_{\lambda\nu}} +
  g^{00} \dot{p} / \rho
  \\ &\approx
  \frac{1}{2} \dv{v^2}{\tau} +
  \frac{1}{2} v^i \dv{v^2}{x^i} +
  U^\nu U^0 \tensor{\Gamma}{^0_{0\nu}} -
  (1 + 2\phi) \dot{p} / \rho
  \\ &\approx
  \frac{1}{2} \dv{v^2}{\tau} +
  \frac{1}{2} v^i \dv{v^2}{x^i} +
  U^\nu U^0 \tensor{\Gamma}{^0_{0\nu}} -
  \dot{p} / \rho
  \\ &\approx
  \frac{1}{2} \dv{v^2}{\tau} +
  \frac{1}{2} v^i \dv{v^2}{x^i} +
  U^\nu U^0 \phi_{,\nu} -
  \dot{p} / \rho
  \\ &\approx
  \frac{1}{2} \dv{v^2}{\tau} +
  \frac{1}{2} v^i \dv{v^2}{x^i} +
  \dot{\phi} + v^i \phi_{,i} -
  \dot{p} / \rho
\end{align*}

(c) A metric is static if there exist coordinates such that $\vec{e}_0$ is timelike, $g_{i0} = 0$, and $g_{\alpha\beta,0} = 0$. Show from Equation 7.6 that a static fluid (i.e. $U^i = 0$, $p_{,0} = 0$, etc) obeys the relativistic equation of hydrostatic equilibrium (Equation 7.40):
%
\begin{displaymath}
  p_{,i} + (\rho + p) \qty[ \frac{1}{2} \ln(-g_{00}) ]_{,i} = 0.
\end{displaymath}

We start by writing out Equation 7.6 as
%
\begin{align*}
  \tensor{T}{^{\mu\nu}_{;\nu}} &=
  [(\rho + p) U^\mu U^\nu]_{;\nu} + [p g^{\mu\nu}]_{;\nu} =
  0
  \\ &=
  [(\rho + p) U^\mu] \tensor{U}{^\nu_{;\nu}} +
  [(\rho + p) U^\nu] \tensor{U}{^\mu_{;\nu}} +
  U^\nu U^\mu (\rho + p)_{,\nu} +
  g^{\mu\nu} p_{,\nu} =
  0
  \\ &=
  \tensor{T}{^{00}_{;0}} +
  \tensor{T}{^{ij}_{;j}} +
  \tensor{T}{^{0i}_{;i}} +
  \tensor{T}{^{i0}_{;0}} =
  0
  \\
  \tensor{T}{^{00}_{;0}} &=
  [(\rho + p) U^0] \tensor{U}{^0_{;0}} +
  [(\rho + p) U^0] \tensor{U}{^0_{;0}} +
  U^0 U^0 (\rho + p)_{,0} +
  g^{00} p_{,0}
  \\ &=
  2 (\rho + p) U^0 \tensor{U}{^0_{;0}}
  \\ &=
  2 (\rho + p) U^0
  [\tensor{U}{^0_{,0}} + U^\lambda \tensor{\Gamma}{^0_{0\lambda}}]
  \\ &=
  2 (\rho + p)
  [U^0]^2 \tensor{\Gamma}{^0_{00}} =
  0
  \\
  \tensor{T}{^{ij}_{;j}} &=
  [(\rho + p) U^i] \tensor{U}{^j_{;j}} +
  [(\rho + p) U^j] \tensor{U}{^i_{;j}} +
  U^j U^i (\rho + p)_{,j} +
  g^{ij} p_{,j}
  \\ &=
  g^{ij} p_{,j}
  \\
  \tensor{T}{^{0i}_{;i}} &=
  [(\rho + p) U^0] \tensor{U}{^i_{;i}} +
  [(\rho + p) U^i] \tensor{U}{^0_{;i}} +
  U^i U^0 (\rho + p)_{,i} +
  g^{0i} p_{,i}
  \\ &=
  [(\rho + p) U^0] \tensor{U}{^i_{;i}} =
  (\rho + p) U^0
  [\tensor{U}{^i_{,i}} + U^\lambda \tensor{\Gamma}{^i_{i\lambda}}]
  \\ &=
  (\rho + p) [U^0]^2 \tensor{\Gamma}{^i_{i0}}
  \\ &=
  \frac{1}{2} [U^0]^2 (\rho + p)
  g^{i\alpha} (g_{\alpha i,0} + g_{\alpha 0,i} - g_{0i,\alpha}) =
  0
  \\
  \tensor{T}{^{i0}_{;0}} &=
  [(\rho + p) U^i] \tensor{U}{^0_{;0}} +
  [(\rho + p) U^0] \tensor{U}{^i_{;0}} +
  U^0 U^i (\rho + p)_{,0} +
  g^{i0} p_{,0}
  \\ &=
  [(\rho + p) U^0] \tensor{U}{^i_{;0}} =
  (\rho + p) U^0
  [\tensor{U}{^i_{,0}} + U^0 \tensor{\Gamma}{^i_{00}}]
  \\ &=
  \frac{1}{2} (\rho + p) [U^0]^2
  g^{i\alpha} (g_{\alpha0,0} + g_{\alpha0,0} - g_{00,\alpha}) =
 -\frac{1}{2} (\rho + p) [U^0]^2 g^{ij} g_{00,j}
  \\ &=
  \frac{1}{2} (\rho + p) g^{ij} g_{00,j} / g_{00} =
  \frac{1}{2} (\rho + p) g^{ij} \ln(-g_{00})_{,j}
  \\
  \tensor{T}{^{\mu\nu}_{;\nu}} &=
  g^{ij} p_{,j} + \frac{1}{2} (\rho + p) g^{ij} \ln(-g_{00})_{,j} = 0
  \\ &=
  p_{,j} + (\rho + p) \qty[ \frac{1}{2} \ln(-g_{00})]_{,j} = 0
\end{align*}


(d) This suggests that there is a relationship between $g_{00}$ and $\exp(2\phi)$ in the case of a static fluid in a Newtonian potential. Show that Equation 7.8 and Exercise 4 are consistent with this.

In the Newtonian limit, the previous equation is unchanged when replacing $ng_{00}$ with $-\exp(2\phi)$, as $\ln(\exp(2\phi))_{,i} = 2 \phi_{,i}$, and
%
\begin{align*}
  \ln(-g_{00})_{,i} &=
  \ln(1 + 2\phi)_{,i} =
  \frac{(1 + 2\phi)_{,i}}{1 + 2\phi}
  \\ &=
  2 \phi_{,i} (1 + 2\phi)^{-1} \approx
  2 \phi_{,i} (1 - 2\phi) \approx
  2 \phi_{,i}.
\end{align*}
%
I'm not really sure how to relate this to Exercise 4, as it relates $\phi_{,\alpha}$ to four-momentum, while this relates it to pressure and density.


\textbf{7}
Consider the (i) Minkowski, (ii) Schwarzschild, (iii) Kerr, and (iv) Robertson--Walker metrics.

(a) Find the conserved components $p_\alpha$ of a the four-momentum of a particle in free-fall.

For this I will use Equation 7.29:
%
\begin{displaymath}
  m \dv{p_\beta}{\tau} = \frac{1}{2} g_{\nu\alpha,\beta} p^\nu p^\alpha.
\end{displaymath}
%
What this tells us is that if $g_{\alpha\beta}$ is independent of $x^\mu$, then $p_\mu$ is constant along the trajectory.

For (i), the metric is independent of all coordinates $(t,x,y,z)$, and so all $p_\alpha$ are conserved.

For (ii), the metric depends on coordinates $r$ and $\theta$, but not $t$ and $\phi$, so only $p_t$ and $p_\phi$ are conserved.

For (iii) we have the same dependencies as (ii).

For (iv) there is an additional time dependence, and so only $p_\phi$ is conserved.


(b) Use the metric for a flat spacetime in spherical polar coordinates to argue that the Schwarzschild and Robertson--Walker metrics are spherically symmetric.

Our metric in (i) can be expressed in spherical polars as
%
\begin{displaymath}
  \dd{s}^2 =
 -\dd{t}^2 + \dd{r}^2 + r^2 (\dd{\theta}^2 + \sin^2\theta \dd{\phi}^2).
\end{displaymath}
%
The Schwarzschild metric can be obtained from this by multiplying $\dd{t}^2$ by $(1 - 2 M / r)$, and dividing $\dd{r}^2$ by it. This newly introduced term only introduces a new radial dependence (the $r^{-1}$ term), not an angular one, so it retains spherical symmetry.

The Robertson--Walker metric can be obtained by dividing $\dd{r}^2$ by $(1 - kr^2)$, and then multiplying everything \emph{except} $\dd{t}^2$ by $R^2(t)$. Again, the $(1 - kr^2)$ term only introduces a radial dependence in its $r^2$ term, and for a given time $t$, $R^2(t)$ is a constant, so spherical symmetry is retained.


(c) For (i') and (ii)--(iv), a geodesic which at one point has $\theta = \pi/2$ and $p^\theta = 0$ (i.e. tangent to the equatorial plane) conserves these quantities. For (i'), (ii), and (iii),use $\vec{p} \cdot \vec{p} = -m^2$ to find $p^r$ as a function of $m$, other conserved quantities, and known functions of position.

(i')
%
\begin{align*}
  \vec{p} \cdot \vec{p} &=
  g_{\alpha\beta} p^\alpha p^\beta =
  g_{\alpha\alpha} (p^\alpha)^2 =
  g_{tt} (p^t)^2 +
  g_{rr} (p^r)^2 +
  g_{\theta\theta} (\cancelto{0}{p^\theta})^2 +
  g_{\phi\phi} (p^\phi)^2
  \\ &=
  -(p^t)^2 + (p^r)^2 + r^2 \cancelto{1}{\sin^2(\theta)} (p^\phi)^2 =
  -m^2
  \\ \implies
  (p^r)^2 &=
  (p^t)^2 - r^2 (p^\phi)^2 - m^2 =
  g^{tt} (p_t)^2 - r^2 g^{\phi\phi} (p_\phi)^2 - m^2 =
  -(p^t)^2 - (p^\phi)^2 - m^2
  \\ \implies
  p^r &=
  \pm\sqrt{-[(p^t)^2 + (p^\phi)^2 + m^2]}
\end{align*}
%
(ii)
%
\begin{align*}
  \vec{p} \cdot \vec{p} &=
  g_{tt} (p^t)^2 +
  g_{rr} (p^r)^2 +
  g_{\phi\phi} (p^\phi)^2
  \\ &=
 -(1 - 2M/r) (p^t)^2 +
  (1 - 2M/r)^{-1} (p^r)^2 +
  r^2 \cancelto{1}{\sin^2\theta} (p^\phi)^2 = -m^2
  \\ \implies
  (p^r)^2 &=
  (1 - 2M/r) [ (1 - 2M/r) (p^t)^2 - r^2 (p^\phi)^2 - m^2 ]
  \\ &=
 -(1 - 2M/r) [ (1 - 2M/r) (p_t)^2 + (p_\phi)^2 + m^2 ]
\end{align*}
%
(iii)
%
This metric gets a bit messy, so I will keep things more abstract. First, I will simplify the metric, utilizing the fact that $\theta = \pi/2$.
%
\begin{gather*}
  \dd{s}^2 =
 -\frac{\Delta - a^2}{r^2} \dd{t}^2 -
  2 \frac{2 M a}{r} \dd{t} \dd\phi +
  \frac{(r^2 + a^2)^2 - a^2 \Delta}{r^2} \dd\phi^2 +
  \frac{r^2}{\Delta} \dd{r}^2 +
  r^2 \dd\theta^2
  \\
  g_{tt} = -\frac{\Delta - a^2}{r^2};
  \quad
  g_{rr} = \frac{r^2}{\Delta};
  \quad
  g_{\theta\theta} = r^2;
  \quad
  g_{\phi\phi} = \frac{(r^2 + a^2)^2 - a^2 \Delta}{r^2};
  \quad
  g_{t\phi} = -\frac{2 M a}{r},
  \\
  \lambda \equiv
  a^6 - 2 (D - r^2) a^4 + (r^4 - 4 M^2 r^2 - 2 D r^2 + D^2) a^2 - D r^4
  \\
  g^{tt} = r^2 (a^4 - (D - 2 r^2) a^2 + r^4) / \lambda;
  \quad
  g^{rr} = \frac{D}{r^2};
  \quad
  g^{\theta\theta} = \frac{1}{r^2};
  \quad
  g^{\phi\phi} = r^2 (a^2 - D) / \lambda;
  \quad
  g^{t\phi} = 2 a M r^3 / \lambda,
  \\
  \vec{p} \cdot \vec{p} =
  g_{tt} (p^t)^2 +
  g_{rr} (p^r)^2 +
  g_{\phi\phi} (p^\phi)^2 +
  2 g_{t\phi} (p^t p^\phi) =
 -m^2
  \\
  p^r =
  \pm\sqrt{
   -g^{rr}
    [g_{tt} (p^t)^2 +
     g_{\phi\phi} (p^\phi)^2 +
     2 g_{t\phi} (p^t p^\phi) +
     m^2]
  }
  \\
  p^t =
  g^{t\alpha} p_\alpha =
  g^{\phi\phi} p_t + g^{t\phi} p_\phi
  \\
  p^\phi =
  g^{\phi\alpha} p_\alpha =
  g^{\phi\phi} p_\phi + g^{t\phi} p_t
\end{gather*}

(d)

When $k = 0$, the line element and metric become
%
\begin{gather*}
  \dd{s}^2 =
 -\dd{t}^2 +
  R^2(t) [ \dd{r}^2 + r^2 (\dd\theta^2 + \sin^2\theta \dd\phi^2) ]
  \\
  g_{tt} = -1;
  \quad
  g_{rr} = R^2(t);
  \quad
  g_{\theta\theta} = R^2(t) r^2;
  \quad
  g_{\phi\phi} = R^2(t) r^2 \sin^2\theta.
\end{gather*}
%
Equation 7.29 with $\beta = r$ then becomes
%
\begin{displaymath}
  m \dv{p_r}{\tau} =
  \frac{1}{2} g_{\nu\alpha,r} p^\nu p^\alpha =
  \frac{1}{2} [ g_{tt,r} (p^t)^2 + g_{rr,r} (p^r)^2 ].
\end{displaymath}
%
Since $g_{tt,r} = g_{rr,r} = 0$, the RHS becomes zero, and so
%
\begin{displaymath}
  m \dv{p_r}{\tau} =
  0 \implies
  \text{$p_r$ is conserved}.
\end{displaymath}



\textbf{8}
For a coordinate system where $g_{\alpha\beta,\mu} = 0$:

(a) Show that $\tensor{T}{^\nu_{\mu;\nu}} = 0$ becomes
%
\begin{displaymath}
  \frac{1}{\sqrt{-g}} (\sqrt{-g} \tensor{T}{^\nu_\mu})_\nu = 0.
\end{displaymath}

For this, I will make mathematicians cry, and go from the \emph{solution} backwards to the starting point. So I expand the final expression, first using the Leibniz rule:
%
\begin{displaymath}
  \tensor{T}{^\nu_{\mu,\nu}} +
  \frac{(\sqrt{-g})_\nu}{\sqrt{-g}} \tensor{T}{^\nu_\mu} =
  0,
\end{displaymath}
%
and then using Equation 6.40:
%
\begin{displaymath}
  \tensor{T}{^\nu_{\mu,\nu}} +
  \tensor{\Gamma}{^\alpha_{\alpha\nu}} =
  0.
\end{displaymath}
%
Just pretend I did that backwards. Next I expand $\tensor{T}{^\nu_{\mu;\nu}}$, to show that the above expression makes it zero.
%
\begin{align*}
  \tensor{T}{^\nu_{\mu;\nu}} &=
  \tensor{T}{^\nu_{\mu,\nu}} +
  \tensor{T}{^\alpha_\mu} \tensor{\Gamma}{^\nu_{\alpha\nu}} -
  \tensor{T}{^\nu_\alpha} \tensor{\Gamma}{^\alpha_{\mu\nu}}
  \\ &=
  \tensor{T}{^\nu_{\mu,\nu}} +
  \tensor{T}{^\nu_\mu} \tensor{\Gamma}{^\alpha_{\nu\alpha}} -
  \tensor{T}{^\nu_\alpha} \tensor{\Gamma}{^\alpha_{\mu\nu}}.
\end{align*}
%
Note that the positive terms are just the expression from before, which we showed was zero, so we're left with
%
\begin{displaymath}
  \tensor{T}{^\nu_{\mu;\nu}} =
  \tensor{T}{^\nu_\alpha} \tensor{\Gamma}{^\alpha_{\mu\nu}}.
\end{displaymath}
%
Now we expand this
%
\begin{align*}
  \tensor{T}{^\nu_{\mu;\nu}} &=
  -\frac{1}{2} \tensor{T}{^\nu_\alpha} g^{\alpha\beta}
  (g_{\beta\mu,\nu} + \cancel{g_{\beta\nu,\mu}} - g_{\mu\nu,\beta})
  \\ &=
  -\frac{1}{2} \tensor{T}{^{\nu\beta}}
  (g_{\beta\mu,\nu} - g_{\mu\nu,\beta}) =
  -\frac{1}{2} \tensor{T}{^{(\nu\beta)}}
  A_{[\nu\beta]\mu} =
  0.
\end{align*}

(b) Suppose $T^{\alpha\beta}$ is zero except in a bounded region of the space-like hypersurface $x^0 =$ constant. Show that Equation 7.41 implies that
%
\begin{displaymath}
  \int_{x^0 = \mathrm{const}} \tensor{T}{^\nu_\mu} \sqrt{-g} n_\nu \dd[3]{x}
\end{displaymath}
%
does not depend on $x^0$, so long as $n_\nu$ is the unit normal to the hypersurface.

Using Equation 7.41 and the differential in Equation 6.18, we take the integral
%
\begin{displaymath}
  \int {
    \frac{1}{\sqrt{-g}} (\sqrt{-g} \tensor{T}{^\nu_\mu})_\nu \sqrt{-g}
  } \dd[4]{x} =
  \int {
    (\sqrt{-g} \tensor{T}{^\nu_\mu})_{,\nu} \dd[4]{x}
  } \dd[4]{x}.
\end{displaymath}
%
Now we use Equation 6.44:
%
\begin{align*}
  \int {
    (\sqrt{-g} \tensor{T}{^\nu_\mu})_{,\nu} \dd[4]{x}
  } \dd[4]{x} &=
  \oint {
    \sqrt{-g} n_\nu \tensor{T}{^\nu_\mu}
  } \dd[3]{S}
  \\ &=
  \int_{x^0 = \mathrm{const}} \sqrt{-g} n_\nu \tensor{T}{^\nu_\mu} \dd[3]{x}
\end{align*}

(c) Now consider flat Minkowski space with a global inertial frame in spherical polar coordinates. Show that, from part (b), we have
%
\begin{displaymath}
  J =
  \int_{t=\mathrm{const}} {
    \tensor{T}{^0_\phi} r^2 \sin\theta
  } \dd{r} \dd\theta \dd\phi,
\end{displaymath}
%
which is independent of $t$. This is the system's total angular momentum.

Since we are in flat Minkowski space, the unit-normal one form has components $\tilde{n} \to (1, 0, 0, 0)$, so only the $\tensor{T}{^0_\mu}$ term is retained. We also have $x^0 \to t$, so we can write the expression from (b) as
%
\begin{displaymath}
  \int_{t = \mathrm{const}} \sqrt{-g} \tensor{T}{^0_\mu} \dd[3]{x}.
\end{displaymath}
%
We also know that $\sqrt{-g} \dd[3]{x}$ in spherical polars is $r^2 \sin\theta \dd{r} \dd\theta \dd\phi$, so we can write this as
%
\begin{displaymath}
  \int_{t = \mathrm{const}} {
    \tensor{T}{^0_\mu} r^2 \sin\theta
  } \dd{r} \dd\theta \dd\phi.
\end{displaymath}
%
Taking the $\phi$ component of $\tensor{T}{^0_\mu}$, we get something which we call $J$:
%
\begin{displaymath}
  J =
  \int_{t = \mathrm{const}} {
    \tensor{T}{^0_\phi} r^2 \sin\theta
  } \dd{r} \dd\theta \dd\phi.
\end{displaymath}

(d) Now express the previous integral in terms of the components of $T^{\alpha\beta}$ on the Cartesian basis, ultimately arriving at
%
\begin{displaymath}
  J =
  \int {
    (x T^{y0} - y T^{x0})
  } \dd{x} \dd{y} \dd{z}
\end{displaymath}

\begin{align*}
  J &=
  \int_{t = \mathrm{const}} {
    \tensor{T}{^0_\phi} r^2 \sin\theta
  } \dd{r} \dd\theta \dd\phi
  \\ &=
  \int_{t = \mathrm{const}} {
    \tensor{\Lambda}{^\alpha_\phi} \tensor{T}{^0_\alpha} r^2 \sin\theta
  } \dd[3]{x}
  \\ &=
  \int_{t = \mathrm{const}} (
    \tensor{\Lambda}{^x_\phi} \tensor{T}{^0_x} +
    \tensor{\Lambda}{^y_\phi} \tensor{T}{^0_y} +
    \tensor{\Lambda}{^z_\phi} \tensor{T}{^0_z}
  ) \dd[3]{x}
  \\ &=
  \int_{t = \mathrm{const}} (
    (-r \sin\theta \sin\phi) \tensor{T}{^0_x} +
    ( r \sin\theta \cos\phi) \tensor{T}{^0_y} +
    (0) \tensor{T}{^0_z}
  ) \dd[3]{x}
  \\ &=
  \int_{t = \mathrm{const}} (
    x \tensor{T}{^0_y} -
    y \tensor{T}{^0_x}
  ) \dd[3]{x}
  \\ &=
  \int_{t = \mathrm{const}} (
    \eta_{yy} x \tensor{T}{^{0y}} -
    \eta_{xx} y \tensor{T}{^{0x}}
  ) \dd[3]{x}
  \\ &=
  \int_{t = \mathrm{const}} (
    x \tensor{T}{^{0y}} -
    y \tensor{T}{^{0x}}
  ) \dd[3]{x}
\end{align*}


\textbf{10}

(a) Show that if the vector field $\xi^\alpha$ satisfies Killing's equation,
%
\begin{displaymath}
  \grad_\alpha \xi_\beta + \grad_\beta \xi_\alpha = 0,
\end{displaymath}
%
then $p^\alpha \xi_\alpha$ is constant along a geodesic.

If $p^\alpha \xi_\alpha$ is constant along a geodesic, then $p^\alpha \xi_{\alpha;\beta} = 0$, so we simply have to show that this follows from Killing's equation.

Killing's equation can be rewritten as
%
\begin{displaymath}
  \xi_{\beta;\alpha} + \xi_{\alpha;\beta} = 0 \implies
  \xi_{\beta;\alpha} = -\xi_{\alpha;\beta}.
\end{displaymath}
%
Now we combine this with the geodesic equation,
%
\begin{displaymath}
  p^\alpha \xi_{\beta;\alpha} = -p^\alpha \xi_{\alpha;\beta} = 0.
\end{displaymath}
%
And there we have it!

(b) Find ten Killing fields for Minkowski spacetime.

Since the basis vectors in Minkowski spacetime are all constant, $\grad_\beta \vec{e}_\alpha = 0$, and so we get four from $\vec{e}_t$, $\vec{e}_x$, $\vec{e}_y$, $\vec{e}_z$. According to part (c), we get a Killing field from any \emph{constant} linear combination of these four, and so from that we may create an infinity of Killing fields. Schutz's solutions manual also lists expressions such as $x \vec{e}_t - t \vec{e}_x$ as Killing fields, which are linear combinations, but the coefficients are non-constant. I give an attempted derivation below, although at the very last step it turns out not to work, and I pretend it does anyway. I claim that the general form of Schutz's expressions is: $x^\alpha \vec{e}_\beta - x^\beta \vec{e}_\alpha$.

\begin{align*}
  \grad_\alpha (x^\alpha \vec{e}_\beta) -
  \grad_\alpha (x^\beta \vec{e}_\alpha) +
  \grad_\beta (x^\alpha \vec{e}_\beta) -
  \grad_\beta (x^\beta \vec{e}_\alpha)
  &=
  \tensor{x}{^\alpha_{;\alpha}} \vec{e}_\beta -
  \tensor{x}{^\beta_{;\alpha}} \vec{e}_\alpha +
  \tensor{x}{^\alpha_{;\beta}} \vec{e}_\beta -
  \tensor{x}{^\beta_{;\beta}} \vec{e}_\alpha
  \\ &=
  \vec{e}_\beta -
  \vec{e}_\alpha -
  \tensor{x}{^\beta_{,\alpha}} \vec{e}_\alpha +
  \tensor{x}{^\alpha_{,\beta}} \vec{e}_\beta
  \\ &=
  \vec{e}_\beta -
  \vec{e}_\alpha -
  \tensor{\Lambda}{^\beta_\alpha} \vec{e}_\alpha +
  \tensor{\Lambda}{^\alpha_\beta} \vec{e}_\beta
  \\ &
  \text{(magnets at work here)}
  \\ &=
  \vec{e}_\beta -
  \vec{e}_\alpha -
  \vec{e}_\beta +
  \vec{e}_\alpha =
  0
\end{align*}


(c) Prove that any \emph{constant} linear combination of two Killing fields $\vec\xi$ and $\vec\eta$ is itself a Killing field.

\begin{align*}
  &
  \grad_\mu \xi_\nu + \grad_\nu \xi_\mu = 0
  \\
  &
  \grad_\mu \eta_\nu + \grad_\nu \eta_\mu = 0
  \\ &
  \grad_\mu (\alpha \xi_\nu + \beta \eta_\nu) +
  \grad_\nu (\alpha \xi_\mu + \beta \eta_\mu)
  \\ =&
  \alpha \grad_\mu \xi_\nu + \beta \grad_\mu \eta_\nu +
  \alpha \grad_\nu \xi_\mu + \beta \grad_\nu \eta_\mu
  \\ =&
  \alpha (\grad_\mu \xi_\nu + \grad_\nu \xi_\mu) +
  \beta (\grad_\mu \eta_\nu + \grad_\nu \eta_\mu) =
  0
\end{align*}


(d)
Show that the Lorentz transforms of the fields in (b) are also Killing fields.

Applying a Lorentz transform $\tensor{\Lambda}{^\mu_\nu}$ we get the expression $\tensor{\Lambda}{^\mu_\nu} \qty(x^\alpha \vec{e}_\beta - x^\beta \vec{e}_\alpha)$.
%
\begin{align*}
  &
  \grad_\alpha
  [
    \tensor{\Lambda}{^\mu_\nu}
    (x^\alpha \vec{e}_\beta - x^\beta \vec{e}_\alpha)
  ] +
  \grad_\beta
  [
    \tensor{\Lambda}{^\mu_\nu}
    (x^\beta \vec{e}_\alpha - x^\alpha \vec{e}_\beta)
  ]
  \\ =&
  \tensor{\Lambda}{^\mu_{\nu;\alpha}}
  [
    (x^\alpha \vec{e}_\beta - x^\beta \vec{e}_\alpha) +
    (x^\beta \vec{e}_\alpha - x^\alpha \vec{e}_\beta)
  ]
  \\ =&
  \tensor{\Lambda}{^\mu_{\nu;\alpha}}
  [
    x^\alpha \vec{e}_\beta - x^\alpha \vec{e}_\beta +
    x^\beta \vec{e}_\alpha - x^\beta \vec{e}_\alpha
  ] = 0
\end{align*}


(e) Use the results in Exercise 7(a) to find Killing vectors for the non-Minkowski metrics listed in (ii)--(iv).

(ii) Since the conserved quantities are $p_t$ and $p_\phi$, then the Killing fields are any constant linear combinations or Lorentz transforms of $\vec{e}_t$, $\vec{e}_\phi$, and $\phi \vec{e}_t - t \vec{e}_\phi$.

(iii) Same as (ii).

(iv) Only $p_\phi$ is conserved, so any constant multiple of $\vec{e}_\phi$ is a Killing field.





\end{document}