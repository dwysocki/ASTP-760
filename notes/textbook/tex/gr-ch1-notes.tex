\documentclass[gr-notes.tex]{subfiles}

\begin{document}

\setcounter{chapter}{0}

\chapter{Special relativity}

\section{Fundamental principles of special relativity (SR) theory}

Special relativity can be summarized by two fundamental postulates:


\begin{enumerate}
\item The principle of relativity (Galileo), which states that no experiment
  may measure the absolute velocity of an observer.
\item The universality of the speed of light (Einstein), which states that
  the speed of light is constant when measured from any inertial reference
  frame.
\end{enumerate}


\section{Definition of an inertial observer in SR}

When we say ``observer'', what we really mean is a coordinate system. Thus an inertial observer is a coordinate system that meets the following 3 criteria:

\begin{enumerate}
\item The distance between two spatial points $P_1$ and $P_2$ is independent of
  time.
\item Time is synchronized and moves at the same rate at all spatial points.
\item At any constant time, space is Euclidean.
\end{enumerate}

It follows from these criteria that the observer must be \textbf{unaccelerated}.

\section{New units}

The speed of light, $c$, is approximately $\SI{3.00e8}{\meter\second^{-1}}$ in SI units. However, these units predate relativity, and are very inconvenient. Life becomes easier if we define our units around $c$, such that $c \equiv 1$.

This can be done by repurposing the meter as a measure of time as well. We thereby define the meter as ``the time it takes light to travel 1 meter''. Thus the speed of light becomes

\begin{displaymath}
  c = \frac{\SI{1}{\meter}}{\SI{1}{\meter}}.
\end{displaymath}

Indeed, it turns out in SR that time is most conveniently measured in distance ($c = \SI{3.00e10}{\centi\meter}$), and in GR mass is as well ($G/c^{-2} = \SI{7.425e-29}{\centi\meter\per\gram}$).


\section{Spacetime diagrams}

\section{Construction of the coordinates used by another observer}

\section{Invariance of the interval}

For two nearby events, we can define the \textbf{invariant interval}, defining a 4D Minskowski spacetime:

\begin{displaymath}
  \dd{s}^2 = -(c \dd{t})^2 + \dd{x}^2 + \dd{y}^2 + \dd{z}^2,
\end{displaymath}

or when we set $c \equiv 1$:

\begin{displaymath}
  \dd{s}^2 = -\dd{t}^2 + \dd{x}^2 + \dd{y}^2 + \dd{z}^2.
\end{displaymath}

This notation can be simplified be defining

\begin{displaymath}
  \eta_{\mu\nu} =
  \mqty(-1 & 0 & 0 & 0 \\
         0 & 1 & 0 & 0 \\
         0 & 0 & 1 & 0 \\
         0 & 0 & 0 & 1);
  \quad
  \dd{s}^2 = \sum_{\mu=0}^3 \sum_{\nu=0}^3 \eta_{\mu\nu} \dd{x}^\mu \dd{x}^\nu
\end{displaymath}



\section{Invariant hyperbolae}

\section{Particularly important results}

\section{The Lorentz transformation}

\section{The velocity-composition law}

\section{Paradoxes and physical intuition}

\section{Further reading}

\section{Appendix: The twin `paradox' dissected}

\section{Exercises}

\textbf{1}
Convert the following to units in which $c = 1$, expressing everything in terms of $\si{\meter}$ and $\si{\kilogram}$.

(Note that
$c = 1 \implies
 1 \approx \SI{3e8}{\meter\per\second}
   \approx (\num{3e8})^{-1}\si{\per\meter\second}$

(a) $\SI{10}{\joule}$
\begin{align*}
  \SI{10}{\joule} &=
  \SI{10}{\newton.\meter} =
  \SI{10}{\kilogram.\meter^2.\second^{-2}} \approx
  \SI{10}{\kilogram.\meter^2.\second^{-2}} \cdot
  ((\num{3e8})^{-1}\si{\meter^{-1}.\second})^2
  \\ &\approx
  \SI{10}{\kilogram} (\num{3e8})^{-2} =
  \SI{10}{\kilogram} \qty(\frac{1}{9} \times 10^{-16}) \approx
  \SI{1.11e-16}{\kilogram}
\end{align*}

(b) $\SI{100}{\watt}$
\begin{align*}
  \SI{100}{\watt} &=
  \SI{100}{\kilogram.\meter^2.\second^{-3}} \approx
  \SI{100}{\kilogram.\meter^2.\second^{-3}} \cdot
  ((\num{3e8})^{-1} \si{\meter^{-1}.\second})^3
  \\ &\approx
  \SI{100}{\kilogram\per\meter} (3^{-3} \times 10^{-24}) =
  \frac{100}{27} \times 10^{-24} \si{\kilogram\per\meter} \approx
  \SI{3.7e-24}{\kilogram\per\meter}
\end{align*}


\textbf{2}
Convert the following from natural units ($c = 1$) to SI units:

(a) A velocity $v = 10^{-2}$.
\begin{displaymath}
  v = 10^{-2} =
  10^{-2} c =
  10^{-2} \SI{3e8}{\meter\per\second} =
  \SI{3e6}{\meter\per\second}
\end{displaymath}

(b) Pressure $P = 10^{19} \si{\kilogram.\meter^{-3}}$.
\begin{align*}
  P &=
  10^{19} \si{\kilogram.\meter^{-3}} \approx
  10^{19} \si{\kilogram.\meter^{-3}} (\SI{3e8}{\meter\per\second})^2
  \\ &\approx
  10^{19} \si{kg.\meter^{-3}} (\SI{9e16}{\meter^2.\second^{-2}}) =
  \SI{9e35}{\newton.\meter^2}
\end{align*}


\textbf{3}
Draw the $t$ and $x$ axes of the spacetime coordinates of an observer $\mathcal{O}$ and then draw:

(a) The world line of $\mathcal{O}$'s clock at $x = \SI{1}{\meter}$.


\end{document}